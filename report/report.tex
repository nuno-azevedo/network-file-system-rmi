\documentclass{article}

\usepackage{geometry}
\geometry{
  a4paper,           % or letterpaper
  textwidth = 14cm,  % llncs has 12.2cm
  textheight = 21cm, % llncs has 19.3cm
  heightrounded,     % integer number of lines
  hratio = 1:1,      % horizontally centered
  vratio = 2:3,      % not vertically centered
}

\usepackage[utf8]{inputenc}
\usepackage[portuguese]{babel}
\usepackage{graphicx, color}
\usepackage{indentfirst}
\usepackage{url}
\usepackage[capposition=top]{floatrow}

\setlength{\parskip}{0.4cm}
\renewcommand{\baselinestretch}{1.1}

\begin{document}

\begin{titlepage}
	\centering
	\def\svgwidth{0.6\textwidth}
	\input{fcup_logo.eps_tex}\par
	\vspace{1cm}
	{\scshape\Large Sistemas Distribuídos\par}
	\vspace{3cm}
	{\Huge\bfseries Network File-System\par}
	\vspace{5cm}
	{\Large\itshape Marco Pontes | Nuno Azevedo\par}
	\vspace{0.5cm}
	{\Large up201308000@fc.up.pt | up201306310@fc.up.pt\par}
	\vfill
	{\large 14 de Novembro de 2016\par}
\end{titlepage}

\newpage
\setcounter{page}{2}

\section{Abstract}

O objetivo deste projeto é a implementação de um sistema de ficheiros em rede (\textit{Network File-System)}. A comunicação cliente - servidor (utilizador - sistema de ficheiros virtual) assenta na tecnologia \textit{Java RMI}\footnote{\url{https://docs.oracle.com/javase/tutorial/rmi/}}. O \textit{Network File-System} suporta várias unidades de armazenamento assim como vários clientes.

\section{Introdução}

Um \textit{Network File-System} é um sistema de ficheiros em rede constituído por uma ou mais unidades físicas de armazenamento, estando disponíveis para o cliente poder guardar os seus ficheiros. Este serviço tem como principal função criar um sistema virtual sobre servidores físicos com o objetivo de armazenar ficheiros de forma distribuída, em computadores interligados por rede.          

De forma a aumentar o espaço disponível no \textit{Network File-System}, pode ser necessário a inserção de mais do que uma unidade de armazenamento mas isso deve ser transparente ao cliente, pois este não tem de se preocupar com a localização real dos seus dados. Com isto, é necessário um servidor que unifique as unidades de armazenamento.

O servidor de meta-dados cria uma camada virtual sobre as unidades de armazenamento de forma a que virtualmente exista apenas um sistema de ficheiros. Sendo assim, este servidor é considerado o núcleo do sistema devido a fazer a ligação entre todos os componentes.

A comunicação entre o cliente, o servidor de meta-dados e as unidades de armazenamento é feita através da tecnologia \textit{Java RMI} sendo este um mecanismo que permite a invocação de métodos residentes em diferentes máquinas virtuais \textit{Java} (\textit{JVM}), permitindo assim ao cliente a manipulação dos ficheiros no servidor remotamente.

\newpage

\section{Descrição}

Após todos os componentes estarem conectados e em sincronia, podemos ver um sistema de ficheiros como o representado na figura \ref{fig:scheme}.\\

\begin{figure}[h]
\def\svgwidth{1\textwidth\centering}
\caption{Interação entre o cliente (\textit{Client}), o servidor de meta-dados (\textit{Meta-data Server}) e as unidades de armazenamento (\textit{Storage Server}).}
\input{scheme.eps_tex}\par
\floatfoot{\textit{Origem:} \url{http://www.dcc.fc.up.pt/~rmartins/aulas/sd1617/trabI/trabI.pdf}\vspace{0.3cm}}
\label{fig:scheme}
\end{figure}

Como referido em cima, pode-se confirmar nesta figura que o servidor de meta-dados \textit{(Meta-data Server)} tem as informações de todas as unidades de armazenamento \textit{(Storage Servers)}.

Do ponto de vista do utilizador, este tem uma vista sobre um sistema de ficheiros de tal forma uniforme, opaco ao conjunto de unidades de armazenamento que o constituem, que tornam a navegação sobre o \textit{Network File-System} semelhante a um sistema de ficheiros local. 

Quanto à implementação, sempre que o cliente \textit{(Client)} precisa de aceder a um ficheiro, tem de contactar o servidor de meta-dados para ficar a conhecer a localização real do ficheiro. Depois disso, pode aceder diretamente à unidade de armazenamento correspondente para efetuar as operações que desejar.

Quando o cliente fizer alguma alteração, a unidade de armazenamento onde essa alteração ocorrer fica responsável por informar o meta-dados do evento para este atualizar as suas informações.
 
\subsection{Metodologia}

Para o desenvolvimento do projeto usamos o \textit{IDE IntelliJ IDEA}\footnote{\url{https://www.jetbrains.com/idea/}} de forma a tornar a programação mais rápida e eficiente, além de que nos ajudou imenso a conhecer os vários métodos disponíveis nas bibliotecas usadas durante o desenvolvimento do projeto.

Pela facilidade de trabalhar em grupo, de ter registos de toda a atividade do projeto, como também nos permitir ter o código guardado de forma segura, usamos a ferramenta \textit{Git}, neste caso o \textit{GitLab\footnote{\url{https://gitlab.com/}}} para manter o projeto privado gratuitamente.

Quanto à divisão das tarefas do projeto, optamos por repartir os métodos de cada classe entre nós, de modo a que cada um conseguisse perceber um pouco de cada componente (\textit{Client, Meta-data Server e Storage Server}).

\subsection{Interfaces/API Usada}

\begin{itemize}
\item \textbf{\textit{Storage Interface}} - Permite que o cliente interaja com um \textit{Storage Server} através dos seguintes métodos:
	\begin{itemize}
		\item \textbf{\textit{create()}:} Cria um novo diretório/ficheiro, ou atualiza um ficheiro já existente.
		\item \textbf{\textit{del()}:} Apaga um diretório/ficheiro. No caso do diretório conter sub-diretórios/ficheiros, este método apaga todos as entradas recursivamente.
		\item \textbf{\textit{get()}:} Retorna o conteúdo do ficheiro da localização indicada na forma de \textit{array} de \textit{bytes}.
	\end{itemize}
\item \textbf{\textit{Meta-data Interface}}
	\begin{itemize}
		\item Métodos invocados pelos \textit{Storage Servers} - Permitem que um \textit{Storage Server} comunique com o servidor de meta-dados informando de todas as alterações que ocorrem:
		\begin{itemize}
			\item \textbf{\textit{addStorageServer()}:} Existência de uma nova unidade de armazenamento.
			\item \textbf{\textit{delStorageServer()}:} Remoção de uma unidade de armazenamento. 
			\item \textbf{\textit{addStorageItem()}:} Criação de um novo diretório/ficheiro.
			\item \textbf{\textit{delStorageItem()}:} Remoção de um diretório/ficheiro.
		\end{itemize}
		\item Métodos invocados pelo \textit{Client} - Permitem fornecer ao cliente informações acerca dos ficheiros armazenados nos vários \textit{Storage Servers}:
		\begin{itemize}
			\item \textbf{\textit{find()}:} Mostra em qual das unidades de armazenamento um determinado item está localizado.
			\item \textbf{\textit{lstat()}:} Lista todos as entradas num determinado diretório ou, no caso do argumento dado ser um ficheiro, mostra o ficheiro se ele existir.
			\item \textbf{\textit{checkExists()}:} Verifica se um determinado diretório/ficheiro existe.
			\item \textbf{\textit{isDir()}:} Verifica se uma determinada entrada é um diretório.
			\item \textbf{\textit{isFile()}:} Verifica se uma determinada entrada é um ficheiro.
		\end{itemize}
	\end{itemize}
\end{itemize}

\subsection{Classes Principais e Suas Funcionalidades}

\begin{itemize}
\item \textbf{StorageServer:} Representa uma unidade de armazenamento responsável pela gestão de ficheiros em disco. Antes de iniciar, verifica se já existem ficheiros no seu diretório local de forma a já os incluir no novo servidor. Efetua as alterações nos diretórios ou ficheiros tal como criar, remover ou alterar e, apenas no caso de ficheiros, enviar ficheiros para o cliente.
\item \textbf{MetaDataServer:} Responsável pela ligação entre todos os \textit{Clients} e o \textit{Storage Servers}. Regista todas as alterações das unidades de armazenamento, incluindo a inserção ou remoção de novas unidades. Fornece ao cliente informações sobre os diretórios/ficheiros existentes nos vários \textit{Storage Servers}.
\item \textbf{Client:} Esta classe proporciona uma interface para o utilizar interagir com o sistema de ficheiros virtual. Conecta-se ao \textit{Meta-data Server} apenas para verificar e obter informações sobre o conteúdo do sistema de ficheiros. Quanto à gestão dos seus ficheiros, o cliente conecta-se aos vários \textit{Storage Servers} para criar e remover diretórios, ou, criar, alterar, remover e mover ficheiros.
\item \textbf{FSTree:} Classe que representa a árvore do sistema de ficheiros virtual utilizada no \textit{Meta-data Server}. Está capacitada para retornar as suas informações, como o nome, tipo, diretório parente e informações sobre os nós filhos de um determinado nó na árvore. Também permite adicionar e remover entradas na árvore, ou seja, criar e remover diretórios e ficheiros.
\item \textbf{Stat:} Desempenha a função de transportar as informações de um determinado nó do sistema de ficheiros, como o seu nome e os seus filhos no caso de ser um diretório, para o cliente, quando este invoca um método para listar as informações de um determinado diretório/ficheiro. 
\end{itemize}


\subsection{Principais Problemas Encontrados e Soluções Propostas}

\begin{itemize}
\item \textbf{Exceções:} Tivemos algumas dificuldades em lidar com todas as exceções lançadas pelos \textit{Storage Servers} e \textit{Meta-data Server} e ter um tratamento adequado para cada uma delas, de modo a que o sistema esteja protegido contra todas as falhas. Por vezes tivemos de propagar exceções lançadas pelo \textit{Meta-data Server}, para um \textit{Storage Server} e esse ter de propagar novamente para um \textit{Client}.
\item \textbf{Caminhos:} Implementámos uma função para fazer \textit{parsing} de todo o tipo de caminhos, sejam eles absolutos, relativos ou até mesmo contendo a sintaxe ``.."\space para indicar o diretório parente. Esta função apesar de não ser nada complexa, demorou algum tempo até funcionar sem problemas, pois devido ao número de possibilidades havia sempre algum caso de falha.
\item \textbf{Unidades de Armazenamento:} Um dos problemas que nos acompanhou desde o início da implementação foi como lidar com um número dinâmico de \textit{Storage Servers}. Quando o cliente se liga ao sistema de ficheiros não tem qualquer conhecimento sobre como este está constituído, e para realizar alguma alteração usando \textit{Java RMI}, é necessário o \textit{stub} da \textit{interface} de cada \textit{Storage}. A solução passou por utilizar o \textit{Meta-data Server} como o fornecedor do \textit{hostname} de cada \textit{Storage} através da função \textit{find()} para depois conseguir fazer um \textit{lookup()} no registo do \textit{RMI} com esse \textit{hostname} e assim conectar-se aos diversos \textit{Storage Servers}.
\item \textbf{Transferência de Ficheiros:} Relativamente a esta ação, a dificuldade que sentimos foi como conseguir transferir ficheiros de qualquer tipo, pois numa implementação inicial o nosso programa apenas suportava a transferência de ficheiros de texto, pois utilizávamos \textit{strings} para representar o conteúdo dos ficheiros. Para transferir por exemplo imagens este método não funcionou, portanto após uma pesquisa mais aprofundada verificámos que a melhor solução seria transferir o ficheiro através de um \textit{array} de \textit{bytes}, para que assim fosse possível transferir qualquer tipo de ficheiro.
\end{itemize}

\section{Conclusão}

De uma forma geral, conseguimos implementar um sistema de ficheiros em rede totalmente funcional, capaz de responder a todos os pedidos de um utilizador.

Não tivemos oportunidade de testar o projeto num ambiente distribuído, no entanto, tivemos sempre a preocupação de a implementação estar preparada para tal, por isso cremos que não existirão muitos problemas para tornar o sistema de ficheiros funcional ao ser composto por diferentes servidores.

A realização deste trabalho permitiu o aprofundamento do nosso conhecimento em vários temas. Nomeadamente, para perceber como um sistema de ficheiros em rede funciona, apesar de que num nível básico, e a forma como realiza a comunicação entre os seus componentes usando o mecanismo \textit{Java RMI}. Permitiu também aprender um pouco mais sobre a programação orientada a objetos.

Consideramos que o tema do trabalho foi adequado para testar a aplicação desta tecnologia e nos deu bastantes ideias para a colocar em prática numa situação real.

\end{document}